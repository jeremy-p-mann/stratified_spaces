\documentclass[11pt,reqno]{amsart}
\usepackage{geometry}                % See geometry.pdf to learn the layout options. There are lots.
\geometry{letterpaper}                    
%\usepackage[parfill]{parskip}    % Activate to begin paragraphs with an empty line rather than an indent
\usepackage{graphicx}
\usepackage{amssymb}
\usepackage{epstopdf}
\usepackage{amsmath}   
\usepackage{amsthm}
\usepackage{amsfonts}
\usepackage{latexsym}
\usepackage{color}
\usepackage{graphicx}
\DeclareGraphicsRule{.tif}{png}{.png}{`convert #1 `dirname #1`/`basename #1 .tif`.png}
%______________
%	 MACROS
%______________



\theoremstyle{plain}
\newtheorem{thm}{Theorem}
\newtheorem{prop}[thm]{Proposition}
\newtheorem{lem}[thm]{Lemma}
\numberwithin{equation}{section}
\newtheorem{cor}[thm]{Corollary}

\theoremstyle{remark}
\newtheorem{rek}[thm]{Remark}
\newtheorem{fact}[thm]{Fact}

\theoremstyle{definition}
\newtheorem{defi}[thm]{Definition}
\newtheorem{ex}[thm]{Example}




%% BTWN PARAGRAPHS
\setlength{\parindent}{2em}
% \setlength{\parskip}{1em}

% FOR THE COMM DIAG
\usepackage{tikz-cd}

 \tikzset{commutative diagrams/.cd,
mysymbol/.style={start anchor=center,end anchor=center,draw=none}
}
\newcommand\MySymb[2][\bigstar]{%
\arrow[mysymbol]{#2}[description]{#1}}
  
%FOR THE FONTS
\usepackage[mathscr]{euscript}

%FOR THE FOOTNOTES
\makeatletter
\def\@xfootnote[#1]{%
  \protected@xdef\@thefnmark{#1}%
  \@footnotemark\@footnotetext}
\makeatother

%COMMANDS
\newcommand{\N}{\mathbb{N}}
\newcommand{\R}{\mathbb{R}}
\newcommand{\Z}{\mathbb{Z}}

\newcommand{\X}{\mathscr{X}}
\newcommand{\Xzero}{\X^{(0)}}
\newcommand{\Xone}{\X^{(1)}}


\newcommand{\Xup}{\X^\uparrow }
\newcommand{\Xright}{\X^\rightarrow }

\newcommand{\xup}{x^\uparrow}
\newcommand{\xright}{x^\rightarrow}


\newcommand{\dX}{\mathrm{d}_\X}

\newcommand{\tright}{\tau_\rightarrow}
\newcommand{\tup}{\tau_\uparrow}
\newcommand{\tdiag}{\tau_\nearrow}

% Link Types
\newcommand{\firstzerolinks}{\mathscr{L}^{(0)}_0}
\newcommand{\firstonelinks}{\mathscr{L}^{(0)}_1}



%% Make TITLE & Begin Documment
\title{The Basic Vocabulary of Stratified Spaces}


%\begin{figure}[h]
%\includegraphics[scale=1]{test}
%\caption{this is a picture}
%\end{figure}


\begin{document}

\maketitle



\section{A Crash Course in Stratified Spaces}

On a conceptual level, the reader is invited to think of a stratified space as a potentially singular manifold, along with a hierarchy of distinguished closed subsets. As we will see shortly, these hierarchies may be conveniently parametrized by a poset. 

\subsection{The Category of Stratified Spaces}


Recall any poset $P$ may be endowed with a topology. Within this topology, a subset $A\subset P$ is open if, for any $a \in A$ and $p \in P$,  if $p \geq a$, then $p \in A$. In other words, open subsets are those which are "closed upwards." Continuous maps between two such posets are precisely those which are order-preserving

\begin{defi}
A \textit{stratified space} is the data of:
\begin{itemize}
\item A paracompact Hausdorff space $X$, and a poset $P$
\item A continuous map $X \rightarrow P$
\end{itemize}
We will refer oto $X$ as the \textit{underlying topological space} of $(X \rightarrow P)$, and $P$ the \textit{stratifying poset}. We will neglect the data of the poset in our notation. 
\end{defi}

\begin{ex}
The map:
\begin{align*}
[0, 1) &\rightarrow \{0 < 1\} \\
0 &\longmapsto 0 \\
0< t &\longmapsto 1
\end{align*}
gives the half open interval the structure of a stratified space. 
\end{ex}
We can extend the above construction to higher dimensions:

\begin{ex}
Geometric simplices stratified spaces:
\begin{align*}
\Delta^p &\rightarrow \{0 < 1 \dots <p \} \\
( t_0, \dots t_n) &\mapsto \mathrm{max}_i \{ t_i \neq 0\}
\end{align*}
Where $\Delta^p = \{ (t_0 , \dots , t_p) \in [0, 1]^{\times p} | \sum t_i =1 \}$. 
\end{ex}

\begin{ex}
The data of a topological space $X$ and a point of $X$, $x \in X$ can be encoded as a stratified space $(X, x)$:
\begin{align*}
X &\longrightarrow \{0 <1\} \\
x &\longmapsto 0 \\
y \in X - \{x\} & \longmapsto 1
\end{align*}
This generalizes to the data of a space with a sequence of increasing closed subsets. Therefore, stratified spaces in the classical sense are stratified spaces in the above sense. The above definition is chosen as allowing more general posets leads to a better behaved category. For example, this category has coproducts. 
\end{ex}


\begin{defi}
A map of stratified spaces $f :(X\rightarrow P) \rightarrow (Y \rightarrow Q)$ is a commutative square of the form:
$$
\begin{tikzcd}
X \arrow[r, "f"] \arrow[d] & Y \arrow[d] \\
P \arrow[r]                & Q          
\end{tikzcd}
$$
Therefore, stratified spaces form a category in the obvious way. 
\end{defi}

\begin{ex}
Given $p \in P$, we obtain an inclusion of stratified spaces:
$$
\begin{tikzcd}
X|_p \arrow[d] \arrow[r, hook] & X \arrow[d] \\
\{p\} \arrow[r, "p"]           & P          
\end{tikzcd}
$$
We will refer to such an $X|_p$ as a \textit{stratum} of $X$
\end{ex}

\subsection{Cones}

Given a partially ordered set $P$, we can formally adjoin a minimal element, $^\triangleleft P$. In the category of posets, this sits in a pushout square:
$$
\begin{tikzcd}
P \times \{0\} \arrow[r] \arrow[d] & P \times \{0<1\} \arrow[d] \\
{0} \arrow[r]                      & ^\triangleleft P          
\end{tikzcd}
$$
\begin{ex}
 $^\triangleleft \{ 1<\dots <n\} \simeq \{0 < 1 \dots < n\}$,
\end{ex}
The next definition extends this construction to stratified spaces:
\begin{defi}
Given a stratified space $(L \rightarrow P)$, the \textit{open cone on} $L$, $C(L\rightarrow P ) = (C(L) \rightarrow C(P)$ is:
\begin{itemize}
\item $C(L)$ is defined to be the pushout of topological spaces: $ * \amalg_{L \times \{0\}} L\times ([0, 1)$, and the stratifying poset to be $C(P) = ^\triangleleft P$
\item The stratifying poset is the map on pushouts induced by:
$$
\begin{tikzcd}
* \arrow[d] & L\times \{0\} \arrow[d] \arrow[l] \arrow[r] & {L \times [0, 1)} \arrow[d] \\
*           & P\times 0 \arrow[l] \arrow[r]               & P \times \{0 < 1\}         
\end{tikzcd}
$$
\end{itemize}
We will refer to $(L\rightarrow P)$ as the \textit{Link}. Note that this sits in a pushout diagram in the category of stratified spaces:
$$
\begin{tikzcd}
L\times \{0\} \arrow[r] \arrow[d] & {L\times [0, 1)} \arrow[d] \\
* \arrow[r]                       & C(L)                      
\end{tikzcd}
$$
\end{defi}

\begin{rek}
Replacing $[0 , 1)$ by $[0, 1]$ in the above definition gives the \textit{closed cone}, $\overline{C}(L)$
\end{rek}

\begin{ex}
Given a finite set $I\rightarrow *$, $C(I)$ is a "graph" with $I$-half edges and a single vertex. Note that $C(I \rightarrow *)$ is different from $C(I \rightarrow I)$, where we are viewing $I$ as a discrete poset. For example, a map $C(I \rightarrow *) \rightarrow (X\rightarrow P)$ must send all of the half edges to the same stratum of $X$.
\end{ex}

\begin{ex}
We can build simplices inductively: 
$$
\overline{C}(\Delta^p \rightarrow \{0 < 1 \dots <p \}) \simeq (\Delta^{p+1} \rightarrow \{0 < 1 \dots < p+1 \})
$$
\end{ex}

\begin{ex}
Note that:
\begin{itemize}
\item $C(S^1) \simeq (\R^2, 0)$
\item $C(*)\times \R \simeq (\R \times \R_{\geq 0}, \R \times \{0\})$
\item $C( S^1 \amalg *)$ looks looks like the union of $\R^2$ and a line through the origin. In particular, it's not a manifold. 
\end{itemize}
\end{ex}

\begin{rek}
Note that the projection map: $L\times [0, 1] \rightarrow [0, 1]$ extends to a map out of the closed cone $C(L)\rightarrow ([0, 1], 0)$
\end{rek}



\subsection{Special Maps}

We now define two classes of maps: open embeddings and refinements. 

\begin{defi}
We'll refer to a map $f (X\rightarrow P) \rightarrow (Y \rightarrow Q)$ as an \textit{open embedding} if:
\begin{itemize}
\item The map $f: X \rightarrow Y$ and $P\rightarrow Q$ are open embeddings of topological spaces
\item The map $f|_p: X_p \rightarrow Y|_{f(p)}$ is an open embedding for every $p \in P$.
\end{itemize}
\end{defi}

\begin{rek}
Note that this notion of an open embedding makes explicit reference to the stratifying poset. 
\end{rek}

\begin{defi}
We'll refer to a map $f (X\rightarrow P) \rightarrow (Y \rightarrow Q)$ as a \textit{refinement} if: 
\begin{itemize}
\item $f: X \rightarrow Y$ is an homeomorphism
\item $f|_p: X_p \rightarrow Y|_{f(p)}$ is a homeomorphism for every $p \in P$.
\end{itemize}
\end{defi}

\begin{ex}
Using the notation in example (-) , the map:
$$
\begin{tikzcd}
{(\R, 0)} \arrow[d] \arrow[r] & \R \arrow[d] \\
\{0<1\} \arrow[r]             & *           
\end{tikzcd}
$$
is a refinement.
\end{ex}

\begin{rek}
A refinement map can be thought of as the "disappearance" of a strata.
\end{rek}


\subsection{$C^0$ stratified spaces}

\begin{defi} 
The category of $C^0$ stratified spaces is the smallest full subcategory of stratified spaces satisfying:
\begin{enumerate}
	\item $\emptyset \rightarrow \emptyset$ and $*\rightarrow *$ are $C^0$ stratified spaces.
	\item If $(L\rightarrow P)$ is a $C^0$ stratified space, the $C(L\rightarrow P)$ is a stratified space
	\item If $X$ is a $C^0$ stratified space, then $X\times \R$ is a $C^0$ stratified space
	\item If $X$ openly embeds in a $C^0$ stratified space, $X$ is a $C^0$ stratified space
	\item If $X$ is covered by a collection $C^0$ stratified spaces, then $X$ is a $C^0$ stratified space.
\end{enumerate}
\end{defi}

\begin{ex}
Note that properties (1), (3), (4), and (5) is a defintion of the category of topological manifolds. Therefore, topological manifolds are $C^0$ stratified space. These are those $C^0$ stratified spaces whose stratifying poset is $*$. Therefore, each strata of a $C^0$ stratified space is a topological manifold. 
\end{ex}

One way to understand $C^0$ stratified spaces is to build it by gluing stratified spaces of the form $C(L)\times \R^n$ along open embedding. 

\begin{defi}
A stratified space of the form:
$$
C(L) \times \R^n
$$
will be referred to as a \textit{basic}
\end{defi}

\begin{rek}
A stratified space is a $C^0$-stratified space if and only if it admits a cover by basics.

 Therefore, every point of a $C^0$-stratified space, $x \in X$ has an open nieghborhood of the form $\overline{C}(L)\times \R^n$. Note that there exists an inclusion: $ L \times \{1\}\times \R^n \rightarrow  \overline{C}(L)\times \R^n$. We will refer to $L\times \R^n$ as the \textit{ link around }$x\in X$.
\end{rek} 

\begin{ex}
let $G = (V, E, s, t)$ be a directed graph in which every point has finite valence and finitely many vertices, where $s, t: E \rightarrow V$ be functions encoding which vertex an edge is attached to. This can be realized as a stratified space of the form $G \rightarrow \{0 < 1 \} $ in the obvious way. The 0-strata are the vertices, while the 1-strata are edges. This is a $C^0$ stratified space because it admits a cover by $C^0$ stratified spaces of the form $\R$ (edges) and $C(I)$, where the cardinality of $I$ coincides with the valency:

Note that a every map of graphs gives a map of $C^0$ stratified spaces of the form: 
$$
\begin{tikzcd}
G_0 \arrow[rd] \arrow[rr] &           & G_1 \arrow[ld] \\
                          & \{0 < 1\} &               
\end{tikzcd}
$$
The fact that the map lies over ${0 < 1}$ enforces the condition that vertices are sent to vertices and edges are sent to edges. 
This example illustrates one of the important features of stratified spaces: they accommodate "singular" objects (which are not too singular). 
\end{ex}

\end{document}
